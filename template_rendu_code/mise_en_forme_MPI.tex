%%%%%%%%%%%%%%%%%%%%%%%%%%%%%%%%%%%%%%%%%%%%
% MISE EN FORME DU TITRE 
%%%%%%%%%%%%%%%%%%%%%%%%%%%%%%%%%%%%%%%%%%%%
\newcommand{\titreRenduCode}[2]{
\noindent
\textsc{Livio Personne (43602)}
\hfill
#2
%\\%[0.4cm]
\begin{center}
\LARGE
\hrule
\vspace{.4cm}
#1 \\[0.2cm]
\textit{Code du TIPE} 
\\[0.4cm]
\hrule
\end{center}
}




%%%%%%%%%%%%%%%%%%%%%%%%%%%%%%%%%%%%%%%%%%%%
% MISE EN FORME DES PIEDS DE PAGE
%%%%%%%%%%%%%%%%%%%%%%%%%%%%%%%%%%%%%%%%%%%%
\AtEndDocument{\label{lastpage}}
\lhead{}
\chead{}
\rhead{}
\lfoot{ 
Code de TIPE - Livio Personne (43602)
}
\cfoot{}
%\rfoot{\thepage/\pageref{lastpage}}
\rfoot{\thepage/21}
\renewcommand{\headrulewidth}{0pt}
\fancyhfoffset{0.7cm}




%%%%%%%%%%%%%%%%%%%%%%%%%%%%%%%%%%%%%%%%%%%%
% MISE EN FORME DU CODE AVEC MINTED
%%%%%%%%%%%%%%%%%%%%%%%%%%%%%%%%%%%%%%%%%%%%

%pour le style des numéros de ligne minted
\renewcommand{\theFancyVerbLine}{\sffamily
\textcolor{expli}{\scriptsize
\oldstylenums{\arabic{FancyVerbLine}}}}

%pour éviter l'italique 
%car apparaissaient en italique à la fois les commentaires et les includes
\newtoggle{inminted}
\AtBeginEnvironment{minted}{\let\itshape\relax}

%paramètres pour le c
\setminted[c]{
	%--fond et cadre	
	%bgcolor = black!3!white,
	frame = leftline,
	framesep = 6pt,
	rulecolor= expli,
	%--numéro de ligne
	linenos=true,
	numbersep=4pt,
	%stepnumber=2,
	xleftmargin=20pt,%pour que les chiffres débordent pas
	%--découpage des longues lignes
	breaklines=true,
	%--tabulations
	tabsize=2,
	}

%paramètres pour le ocaml
\setminted[ocaml]{
	%--fond et cadre	
	%bgcolor = black!3!white,
	frame = leftline,
	framesep = 6pt,
	rulecolor= expli,
	%--numéro de ligne
	linenos=true,
	numbersep=4pt,
	%stepnumber=2,
	xleftmargin=20pt,%pour que les chiffres débordent pas
	%--découpage des longues lignes
	breaklines=true,
	%--tabulations
	tabsize=2,
	}

%paramètres pour le python
\setminted[python]{
	%--fond et cadre	
	%bgcolor = black!3!white,
	frame = leftline,
	framesep = 6pt,
	rulecolor= expli,
	%--numéro de ligne
	linenos=true,
	numbersep=4pt,
	%stepnumber=2,
	xleftmargin=20pt,%pour que les chiffres débordent pas
	%--découpage des longues lignes
	breaklines=true,
	%--tabulations
	tabsize=2,
	}


