\documentclass[12pt,a4paper,fleqn]{article}

%%%%%%%%%%%%%%%%%%%%%% LES PACKAGES
\input{packages_MPI.tex}

%%%%%%%%%%%%%%%%%%%%%% MISE EN FORME
\colorlet{expli}{gray}
\input{mise_en_forme_MPI.tex}

%%%%%%%%%%%%%%%%%%%%%%%%%%%%%%%%%%%%%%%
\begin{document}
\pagestyle{fancy} %active les pieds de pages


%%%%%%%%%%%%%%%%%%%%%%%%%%%%%%%%%%%%%%%
\titreRenduCode{Détection et repérage d'un ballon de volley-ball à partir d'une entrée vidéo}{20 Juillet 2024}
%%%%%%%%%%%%%%%%%%%%%%%%%%%%%%%%%%%%%%%

%\input{conseils.tex} % ligne à commenter pour votre rendu 


\tableofcontents
\chapter{}


% Fichiers à entrer
% ../color.py ----------------------------
% ../circle.py ---------------------------
% ../reward_functions.py -----------------
% ../find_ball_v6.py ---------------------
% ../classifieur.py ----------------------
% ../track.py ----------------------------
% ../compteur_de_balle.py ----------------


%%%%%%%%%%%%%%%%%%%%%%%%%%%%%%%%%%%%%%%
\newpage
\section{Fichiers auxiliaires}

%================================= color.py
\subsection{Code du fichier \mintinline{text}{color.py}}
\inputminted{python}{../color.py}

%================================= circle.py
\subsection{Code du fichier \mintinline{text}{circle.py}}
\inputminted{python}{../circle.py}

%================================= reward_function.py
\newpage
\subsection{Code du fichier \mintinline{text}{reward_function.py}}
\inputminted[firstline=1,lastline=23]{python}{../reward_functions.py}

%================================= compteur_de_balle.py
\subsection{Code du fichier \mintinline{text}{compteur_de_balle.py}}
\inputminted[firstline=1,lastline=57]{python}{../compteur_de_balle.py}



%%%%%%%%%%%%%%%%%%%%%%%%%%%%%%%%%%%%%%%%%%
\newpage
\section{Arbres k-dimensionnels: code du fichier \mintinline{text}{classifieur.py}}

\subsection{Classe de donnée}
\inputminted[firstline=1,lastline=36]{python}{../classifieur.py}
\subsection{Intéractions avec un fichier csv}
\inputminted[firstline=42,lastline=147]{python}{../classifieur.py}
\subsection{Intéractions avec un arbre k-dimensionnel}
\inputminted[firstline=155,lastline=239]{python}{../classifieur.py}
\newpage
\subsection{Fonctions d'affichage}
\inputminted[firstline=246,lastline=307]{python}{../classifieur.py}


%%%%%%%%%%%%%%%%%%%%%%%%%%%%%%%%%%%%%%%%
\newpage
\section{Trajectoires: code du fichier \mintinline{text}{track.py}}

\subsection{Fonctions auxiliaires}
\inputminted[firstline=1,lastline=17]{python}{../track.py}
\subsection{Classe de trajectoire}
\inputminted[firstline=19,lastline=126]{python}{../track.py}


%%%%%%%%%%%%%%%%%%%%%%%%%%%%%%%%%%%%%%%%%
\newpage
\section{Fichier principal: code du fichier \mintinline{text}{find_ball_v6.py}}

\subsection{Fonctions d'analyse de contours}
\inputminted[firstline=1,lastline=142]{python}{../find_ball_v6.py}
\subsection{Variables globales et déclenchement du main}
\inputminted[firstline=285,lastline=295]{python}{../find_ball_v6.py}
\newpage
\subsection{Fonction principale d'analyse vidéo}
\inputminted[firstline=146,lastline=282]{python}{../find_ball_v6.py}



\end{document}
